%
% ここから、各章毎に分割したソースファイルを順番に読込
%
\begin{document}
  %
  \maketitle                    % 表紙
  %\input{101-abstract}          % はじめに
  %\input{111-index-0}           % 目次
  %%%
  %%%
  %\input{121-index-1}           % 
  %\input{131-self-introduce}    % 自己紹介
  %%%
  %\input{201-index-2}           % 
 % \input{211-introduction}      % 
  %%%
  %\input{301-index-3}           % 
  %\input{311-example-probrem}   % 本日の例題と以前の勉強会で報告された結果
  %\input{312-previous-comments} % 以前の勉強会でのコメント
  %%%
  %\input{401-index-4}           % 改めて解いてみた
  %\input{411-index-4B}          % 改めて解いてみた
  %\input{412-improvements}      % 今回の改善点
  %\input{421-index-4C}          % 改めて解いてみた
  %\input{422-meshing}           % 改めてメッシュについて復習
  %\input{423-meshing2}          % 改めてメッシュについて復習
  %\input{431-index-4D}          % 
  %\input{432-som}               % 今回使用するソフト
  %\input{441-index-4E}          % 
  %\input{442-create-solid}      % 形状作成
  %\input{443-create-solid2}     % 形状作成
  %\input{444-create-solid3}     % 形状作成
  %\input{451-import-solid}      % 形状取り込み
  %\input{461-create-mesh1}      % メッシュ切
  %\input{462-create-mesh2}      % メッシュ切
  %\input{471-create-boundary}   % 境界条件作成
  %\input{481-create-etc}        % もろもろ作成
  %\input{482-results00}         % 結果0
  %%%
  %\input{501-index-5A}          % 改めて解いてみた
  %\input{511-results01}         % 結果1
  %\input{512-results02}         % 結果2
  %\input{521-index-5B}          % 改めて解いてみた
  %\input{522-results04}         % 結果3
  %\input{523-results05}         % 結果4
  %\input{524-results06}         % 結果5
  %%%
  %\input{701-index-5}           % 
  %\input{711-conclusions}       % まとめ
  %%%
  %\input{801-references}        % 参考文献
  %%%
  \begin{frame}[noframenumbering]{}
	ご清聴、ありがとうございました
\end{frame}
           % ご清聴、ありがとうございました
  %\input{911-appendices-A}      % 付録 ~今回使用したソフト~
  %\input{912-appendices-B}      % 付録 ~配布するデータファイル~
  %\input{913-appendices-C}      % 付録 ~ソースのありか~
\end{document}
